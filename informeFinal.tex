\documentclass[a4paper]{article}
\usepackage[utf8]{inputenc}
\usepackage{amsmath,amssymb,latexsym}
\usepackage{syntax}
\usepackage[margin=1.5cm]{geometry}
\usepackage{amssymb}
\usepackage{tipa}
\usepackage{listings}
\usepackage{graphicx}
\usepackage{stmaryrd}
\usepackage{setspace}

\setlength{\grammarindent}{2cm}

\lstset{
  basicstyle=\itshape,
  xleftmargin=3em,
  literate={:=}{$\rightarrow$}{2}
           {α}{$\alpha$}{1}
           {δ}{$\delta$}{1}
}

\linespread{1.3}

\author{
        Dzikiewicz, Luis\\
        Legajo: D-3850/4\\
        \texttt{luisdzi.87@gmail.com}
        \and Ernandorena, Iván\\
        Legajo: E-1115/1\\
        \texttt{ivan.ernandorena@gmail.com}
}

\date{}

\title {
    \Huge  \textsc{Trabajo Práctico 4\\}
    \large \textsc{Teoría de Base de Datos}
}

\begin{document}


    \pagenumbering{gobble}
    \maketitle

    \thispagestyle{empty}

    \begin{center}
         \large \bf Docentes
    \end{center}

    \begin{center}

        Claudia Deco  

        Cristina Bender

        Lucía Reixach 
        \vspace{2cm}

        \includegraphics[scale=1.5]{Logo-Unr}


    \end{center}




%------------------------------------------------------------------------------


    \newpage


    \section*{Ejercicio 3:}
    \begin{verbatim}
    res := empty;
    pr  := P(R);
    while(pr no sea vacio)
      for each alfa in pr
        pr := pr - alfa;
        if alfa+ == R then agregar alfa a resultado;
        for each beta in pr
          if alfa esta en beta then pr := pr - beta;  
    \end{verbatim}

    \begin{verbatim}
    res = empty;
    pr  = partes(R);
    while(pr != empty){
      for(alfa in pr){
        pr := pr - alfa;
        if alfa+ == R then union alfa resultado;
        for(beta in pr){
          if alfa isSubsetOf beta then pr := pr - beta;
        }
      }
    } 
    \end{verbatim}


\end{document}
